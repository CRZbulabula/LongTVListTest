\documentclass[UTF8]{article}
\usepackage{CTEX}

\begin{document}

\section{实验参数设置}
\subsection{LinkedList}
LinkedList中的数据为long类型的单值

\subsection{SkipList}
SkipList中的数据为形如$<long, long>$的键-值对,插入一个新节点时,节点向上分裂的概率为0.5

\subsection{BPlusTree}
BPlusTree中的数据为形如$<long, long>$的键-值对,BPlusTree的阶数为1024

\subsection{LongTVList}
LongTVList中的数据为形如$<long, long>$的时间戳-值对,底层链表的长度不超过128

\section{顺序插入结果}
将$1$到$n$的数据顺序插入数据结构,统计插入时间

\begin{tabular}{|c|c|c|c|c|}

    \hline
     & $10^3$ & $10^4$ & $10^5$ & $10^6$ \\
    \hline
    LinkedList & 0.2ms & 0.7ms & 1.7ms & 32.6ms\\
    \hline
    SkipList & 0.5ms & 1.9ms & 14.5ms & 172.6ms\\
    \hline
    BPlusTree & 0.9ms & 5.3ms & 15.5ms & 120.2ms\\
    \hline
    LongTVList & 0.4ms & 1.0ms & 3.6ms & 22.1ms\\
    \hline

\end{tabular}

\section{随机插入结果}
将$1$到$n$的一个排列插入数据结构,统计插入及维护数据有序的总耗时

\begin{tabular}{|c|c|c|c|c|}

    \hline
     & $10^3$ & $10^4$ & $10^5$ & $10^6$ \\
    \hline
    LinkedList & 12.0ms & 293.0ms & 34929ms & 21311685ms\\
    \hline
    SkipList & 0.5ms & 2.4ms & 32.4ms & 779.2ms\\
    \hline
    BPlusTree & 1.0ms & 6.4ms & 42.0ms & 825.7ms\\
    \hline
    LongTVList(insert) & 0.2ms & 0.6ms & 2.6ms & 18.4ms\\
    \hline
    LongTVList(sort) & 1.7ms & 7.8ms & 60.5ms & 747.4ms\\
    \hline
    LongTVList(total) & 1.9ms & 8.4ms & 63.1ms & 765.8ms\\
    \hline

\end{tabular}

\section{插入时间分析}
实验结果表明,在顺序插入时B+树表现比跳表更好,而随机插入时,跳表表现得比B+树更好,理论分析如下:

新插入一个跳表中的节点,该节点将不断以0.5的概率向上扩展,因此每一层跳表中的节点数期望是下一层的$\frac 12$。
在跳表中插入一个新节点,将从顶层开始,依次遍历每层节点,确定位置后转入下一层。这个过程实际上与二分查找等价,因此跳表中插入一个新节点的复杂度始终为$O(\log_2 n)$。

新插入一个B+树中的节点,当插入数据为升序时,因为新数据总比之前所有数据都大,所以新数据总是插入在当前节点的最后一个子节点所在的子树内,这个特性可以让当前节点省去二分查找,直接确定新节点应该属于的子节点的位置。
因此节点度为1024的B+树在插入顺序数据时的效率能达到$O(\log_{1024} n)$。
对于随机数据插入,B+树需要在每一层二分查找该数据需要插入的子节点的位置,复杂度为$O(\log_{2} 1024)$,此外,B+树的层数为$O(\log_{1024} n)$。
因此当插入随机数据时,B+树的插入复杂度为$O(\log_{2} 1024) \times O(\log_{1024} n) = O(\log_{2} n)$,与跳表的理论复杂度相当。
但是B+树的插入过程是双层二分查找,常数复杂度高于跳表的普通二分查找,而且B+树在插入节点后具有比跳表更复杂的向上分裂过程,故当插入随机数据时,B+树的表现将不如跳表。

\end{document}